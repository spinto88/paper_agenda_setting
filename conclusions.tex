\section{Conclusions}

\par The study of Mass Media, and in particular the agenda setting theory, can be empowered by the used of data mining or machine learning algorithms. 
In this work, through the implementation of a topic detection algorithm we could describe the Agenda of the Media as a distribution which evolves over time and which is defined in a topic's space which emerges from the analysis of the corpus.
This gave us an insight of how we can construct and follow the Public's interests, the Public Agenda, in order to compare with the Media Agenda, i.e. Media interests. 
\par Given the Agendas, we found that the Public one is usually less diverse than the Media, showing that when there is a very attractive topic, the audience focus on this one, when the Media has to cover the other too. 
On the other hand, the measurement of distances between Agendas can be employed to rapidly detect periods when the Public may have an independent behavior respect to the Media.
\par The methodology implemented allow us to detect coverage bias in newspapers and gave us a first approximation in the theory of framing. 
\par We hope that some of the elements studied here will give us insights at the time of proposing a mathematical model about Mass Media and Public interaction. Future works may include a more systematic study and its extension to international Media, a deeper study of framing through topic detection and sentiment analysis, and the inclusion of social media in our analysis.
