\subsection{Social Media}

\par The interaction between media and public is rather complex, and the advance of social media gives new direction in the exploration of this aspect of agenda-setting theory. 
\par In \cite{russell2014dynamics} the question of \emph{causality} is faced up. Their study shows that sometimes the traditional media set the agenda and sometimes, the social media does. They show that social media is always more interested in social issues than the traditional one, and despite the existence of correlation, the social media agenda can not be seen as a \emph{slave} of the traditional media. 
\cite{soroka2017negativity} shows that the newspapers and Twitter have an opposite reaction to the changes of the unemployment rates, and for instance, in Argentina, \cite{mitchelstein2016brecha} shows that not necessarily the journalists and public's preferences coincide, by studying the most viewed articles and the home page articles in online news sites.
\par Other works study various aspects of social media, such as the danger of selective exposure \cite{feezell2017agenda}\cite{messing2014selective}\cite{bakshy2015exposure} and the role of media organization in Twitter discussions \cite{calvo2016time}\cite{malik2016macroscopic}.


