
\subsection{Agenda Setting Theory}

% Introducción y descripción de agenda-setting en su nivel más básico.
\par In the famous study performed in Chapel Hill during the US presidential elections in 1968 \cite{mccombs1972agenda}, Maxwell McCombs and Donald Shaw found that those aspects of public affairs that are prominent in the news become prominent among the public.
This study is considered the founding of the agenda-setting theory, which focus in the influence of mass media in public opinion.  
From \cite{mccombs2014agenda}, \textit{``The media agenda is the pattern of news coverage over a period of days, weeks (...) for a set of issues or other topic. In other words, the media agenda is a systematic compilation of the issues or topics presented to the public that identifies the degree of emphasis on these topics.''}

\par Since the Chapel Hill research, several directions of agenda-setting were established \cite{mccombs2005look}.
In its basic stage, the theory is focus on the comparison between the topics coverage by the media and the public agenda, i.e. the topics that the public consider as priority.
It looks for answering the question if the media is able to set the public agenda, which would transform the media as an important actor in the formation of public opinion. The interaction between media and public is rather complex, for instance, \cite{mitchelstein2016brecha} shows that not necessarily the journalists and public preferences coincide.
\par With the emergency of the Internet, the end of agenda-setting were predicted due to the audience fragmentation onto multiple sources, which would virtually lead to a highly individualized agenda.
However, it is based on two assumptions that not necessarily are true: that the public spreads its attention in an homogeneous way across the multiple sources, and that the agendas of that sources are different \cite{mccombs2005look}.

% Agenda setting de segundo nivel: noticias con atributos y contextualización.
\par The basic agenda-setting sometimes is called the ``first level agenda-setting''. 
The very often quoted phrase of Bernard C. Cohen \textit{``The press may not be successful much of the time in telling people what to think, but it is stunningly successful in telling its readers what to think about.''} illustrates its object of study.
On the other hand, the ``second level agenda-setting'', sometimes called \textit{attribute agenda-setting}, studies the \textit{objects} (in a social psychology way, where an \textit{attitude object} designate a thing that an individual has an attitude or opinion about) present in the media agenda. When the media talks about an object some attributes are emphasized, and others not. 

\par The ``second level agenda-setting'' is linked with \textit{framing} \cite{guggenheim2015dynamics} \cite{tsur2015frame}. 
To frame is to \textit{select some aspects of a perceived reality and make them more salient in a communicating text, in such a way as to promote a particular problem definition, causal interpretation, moral evaluation and/or treatment recommendation} (Robert Entman) \cite{mccombs2005look}.
This stage of agenda-setting theory can be summary in the phrase \textit{``the media not only can be successful in telling us what to think about, they also can be successful in telling us how to think about it.''} 

% Intermedia setting-agenda: compentencia entre medios de comunicación.
\par Other interesting stage of agenda-setting concerns with the sources of media agenda, i.e. if the media set the public agenda, \textit{who sets the agenda media?} 
Within this framework, \textit{intermedia agenda-setting} observes the competition between different media and how they influence each other.
The competition between mass media for the same audience can lead to a homogenization of the agendas \cite{vargo2017networks}, which is in the opposite direction of one of the assumptions that predict the end of agenda-setting, as was mentioned before.

