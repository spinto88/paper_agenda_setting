
\begin{abstract}

The agenda-setting theory is a practical framework in order to understand the role of mass media on a society. The theory treats mass media as a very important actor that is able to make people think about, and in many cases how to think about certain topics.  
When the media successes in this task, we say that the media \textit{set the
agenda.}
In this work we study the agenda of Argentinian newspapers in comparison with public's interests through a quantitative approach by performing topic detection over the news, identifying the main topics covered and their evolution over time.We look for characterizing the differences and similarities over time between what what we call the Media Agenda and the Public Agenda.
On the other hand, we aim to detect coverage bias among the newspapers involved in the analysis in the emerging topics. \\

\end{abstract}


