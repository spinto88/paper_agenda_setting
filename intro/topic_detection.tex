\subsection{Topic detection}

\par Unsupervised topic modeling is a widely used technique in the analysis of a large corpus of documents. 
It is an alternative to the dictionary-based analysis, which is the most popular automated analysis approach in social science research \cite{guo2016big}, and allows to work with a corpus without a prior knowledge, letting the topics emerge from the data. Despite the popularity of this methods, we believe that there is still a lack in the employment of these ones through the lens of the agenda setting framework.
\par As mentioned above, many works based on news corpus emphasize the performance of the topic model over a labeled corpus, focusing on the proper detection of a given topic \cite{dai2010online}\cite{po2016topic}\cite{brun2000experiment} and without taking into account temporal information. 
The temporal profile of topics is usually embedded in the context of topic tracking \cite{hu2016news}\cite{li2017joint}, or in the recognition of emerging topics in real-time \cite{cataldi2010emerging} mostly applied to social media.




